\documentclass[12pt]{article}
\usepackage{graphicx}
\usepackage{xcolor}
\usepackage{hyperref}
\usepackage{wasysym}
\usepackage{mathrsfs}
\usepackage[calc]{datetime2}
\usepackage{boondox-cal}
\usepackage{bm} % For bold math symbols
\usepackage{amsmath}
\usepackage{amssymb}
\usepackage{cancel}

\DTMsavenow{now}
\DTMsavedate{DueDate}{2024-01-23}

\newcount\daystilldue
\newcount\hourstilldue
\newcount\minutestilldue

% Calculate the difference in days
\DTMsaveddatediff{DueDate}{now}{\daystilldue}

% Calculate the difference in hours and minutes
\hourstilldue=\DTMfetchhour{DueDate}
\advance\hourstilldue by -\DTMfetchhour{now}
\minutestilldue=\DTMfetchminute{DueDate}
\advance\minutestilldue by -\DTMfetchminute{now}

% Adjust for negative values
\ifnum\hourstilldue<0
  \advance\hourstilldue by 24
  \advance\daystilldue by -1
\fi
\ifnum\minutestilldue<0
  \advance\minutestilldue by 60
  \advance\hourstilldue by -1
\fi

\newcommand{\TimeUntilDue}{
  \ifnum\daystilldue<4
    \textcolor{red}{
    \number\daystilldue\ days - 
    \number\hourstilldue\ hours - 
    \number\minutestilldue\ min until deadline!!
  }
\else
    \number\daystilldue\ days - 
    \number\hourstilldue\ hours - 
    \number\minutestilldue\ min until deadline
  \fi
}


\begin{document}

\title{ECE 6310 - Advanced Electromagnetic Fields: Homework Set \#1}
\author{Miguel Gomez}
%\date{\TimeUntilDue}
\maketitle

% Problem 1.3
\section{Preliminaries}

In this document, we use standard notation for electromagnetic theory. Key equations and concepts are summarized below:

\subsection*{Vector Notation}
\begin{itemize}
  \item $\bm{\mathcal{E}}$: Electric field intensity
  \item $\bm{\mathcal{H}}$: Magnetic field intensity
  \item $\bm{\mathcal{D}}$: Electric flux density
  \item $\bm{\mathcal{B}}$: Magnetic flux density
  \item $\bm{\mathcal{J}}$: Current density
  \item $\rho_v$: Volume charge density
\end{itemize}

\subsection*{Differential Operators}
\begin{itemize}
  \item $\nabla \cdot\ \ $: Divergence of a vector field
  \item $\nabla \times\ \ $: Curl of a vector field
  \item $\nabla\ \ $: Gradient of a scalar field
  \item $\partial_i\ \ $: Partial derivative with respect to the independent basis element $i$
\end{itemize}

\subsection*{Maxwell's Equations}
In integral form, Maxwell's equations are given by:
\begin{align}
  \oint_{\partial V} \bm{\mathcal{E}} \cdot d\bm{\mathcal{l}} &= - \frac{d}{dt} \int_{V} \bm{\mathcal{B}} \cdot d\bm{\mathcal{S}} & \text{(Faraday's Law of Induction)} \\
  \oint_{\partial V} \bm{\mathcal{H}} \cdot d\bm{\mathcal{l}} &= \int_{V} \bm{\mathcal{J}} \cdot d\bm{\mathcal{S}} + \frac{d}{dt} \int_{V} \bm{\mathcal{D}} \cdot d\bm{\mathcal{S}} & \text{(Ampère's Circuital Law)} \\
  \oiint_{\partial V} \bm{\mathcal{D}} \cdot d\bm{\mathcal{S}} &= \int_{V} \rho_v dV & \text{(Gauss's Law for Electricity)} \\
  \oiint_{\partial V} \bm{\mathcal{B}} \cdot d\bm{\mathcal{S}} &= 0 & \text{(Gauss's Law for Magnetism)}
\end{align}

\subsection*{Other Relevant Equations}
\begin{itemize}
  \item Continuity Equation: $\nabla \cdot \bm{\mathcal{J}} + \partial_t \rho_v = 0$
  \item Relationship between $\bm{\mathcal{E}}$, $\bm{\mathcal{D}}$: $\bm{\mathcal{D}} = \epsilon \bm{\mathcal{E}}$
  \item Relationship between $\bm{\mathcal{H}}$, $\bm{\mathcal{B}}$: $\bm{\mathcal{B}} = \mu \bm{\mathcal{H}}$
\end{itemize}

\subsection*{Boundary Conditions}
Discuss the boundary conditions for $\bm{\mathcal{E}}$, $\bm{\mathcal{H}}$, $\bm{\mathcal{D}}$, and $\bm{\mathcal{B}}$ at interfaces between different media.

\section*{Problem 1.3}
The electric flux density inside a cube is given by:
\begin{itemize}
    \item[(a)] \( \vec{D} = \hat{a}_x (3 + x) \)
    \item[(b)] \( \vec{D} = \hat{a}_y (4 + y^2) \)
\end{itemize}
Find the total electric charge enclosed inside the cubical volume when the cube is in the first octant with three edges coincident with the x, y, z axes and one corner at the origin. Each side of the cube is 1 m long.
\newline
Ok, this one we can do by using the following expressions from the text \cite{balanis_2012}:
\begin{equation}\label{stokes}
  \oiint_{s}\bm{\mathcal{D}}ds = \iiint_{v} \mathcal{q}_{ev}dv =  \mathscr{Q}_{e}
\end{equation}
\begin{equation}\label{deldot}
  \mathcal{q}_{ev} =  \Delta \cdot \bm{\mathcal{D}}   
\end{equation}

We apply expression \ref{deldot} then plug into expression \ref{stokes} to get the total electric charge.
\begin{equation*}
  \iiint_{V}\mathcal{q}_{ev} = \iiint_{V}\Delta \cdot (\hat a_x (3+x))dV = \iiint_{V} \partial_x (\hat a_x (3+x))dV= \iiint_{0}^{1}1dV= 1 
\end{equation*}
\begin{equation*}
    \iiint_{V}\mathcal{q}_{ev} = \iiint_{V}\Delta \cdot (\hat a_y (4+y^2))dV =  \iiint_{V} \partial_y (\hat a_y (4+y^2))dV = \iiint_{0}^{1}2y\ dV
\end{equation*}
\begin{equation*}
\iiint_{0}^{1}2y\ dV = \iint_{0}^{1}\int_{0}^{1}2ydy\ ds = \iint_{0}^{1}y^2|_{0}^{1} ds =  \iint_{0}^{1}(1-0)\ ds = 1
\end{equation*}
% Problem 1.4
\section*{Problem 1.4}
An infinite planar interface between media, as shown in the figure, is formed by having air (medium \#1) on the left of the interface and lossless polystyrene (medium \#2) to the right of the interface. An electric surface charge density \( \sigma_{es} = 0.2 \, \text{C/m}^2 \) exists along the entire interface. The static electric flux density inside the polystyrene is given by \( \vec{D}_2 = 6\hat{a}_x + 3\hat{a}_z \, \text{C/m}^2 \).
Determine the corresponding vector:
\begin{itemize}
\item[(a)] Electric field intensity inside the polystyrene. \\
  we can find this using the following expression for the Electric field intensity vector:
  \begin{align*}
    \epsilon_{poly} &= 2.56\epsilon_0 \\
    \bm{\mathcal{D}} &= \epsilon\bm{\mathcal{E}}\\
    \bm{\mathcal{E}} &= \epsilon^{-1}\bm{\mathcal{D}} = 0.390625\epsilon_0^{-1}\cdot\bm{\mathcal{D}} = \epsilon_0^{-1} (2.34\cdot a_x\ +\ 1.17\cdot a_z)
  \end{align*}
\item[(b)] Electric polarization vector inside the polystyrene.\\ 
  we can find this using the following expression for the polarization vector:
  \begin{align*}
    \bm{\mathcal{P}}_e &= \epsilon_0\chi_e\bm{\mathcal{E}}\\
    \chi_e = \epsilon_r- 1 &= 2.56 - 1 = 1.56 \\
    \bm{\mathcal{P}}_e = 1.56\textcolor{red}{\cancelto{\textcolor{black}{1}}{\textcolor{black}{\epsilon_0\epsilon_0^{-1}}}} (2.34\cdot a_x\ &+\ 1.17\cdot a_z) = 3.6504\cdot a_x\ +\ 1.8252\cdot a_z 
  \end{align*}
\item[(c)] Electric flux density inside the air medium.\\ 
  By the continuity of tangential components at the boundary: \\
  $\mathbf{\hat n}\times(\bm{\mathcal{E}}_2 - \bm{\mathcal{E}}_1) = 0$\\
  we apply the following.
  \begin{align*}
    \bm{\mathcal{E}}_{air} &= \bm{\mathcal{E}}_{Poly}\\
    \bm{\mathcal{D}}_{air} &= \epsilon_{air}\bm{\mathcal{E}}_{air}\\
    \therefore \bm{\mathcal{D}}_{air}\epsilon_{air}^{-1} &=  \bm{\mathcal{D}}_{Poly}\epsilon_{Poly }^{-1}\\
    \bm{\mathcal{D}}_{air} &= \frac{\epsilon_{air}}{\epsilon_{Poly}}\bm{\mathcal{D}}_{Poly} = \frac{\textcolor{red}{\cancel{\textcolor{black}{\epsilon_0}}}\cdot 1}{\textcolor{red}{\cancel{\textcolor{black}{\epsilon_0}}}\cdot 2.56}\cdot \bm{\mathcal{D}}_{Poly}\\
    &= \frac{6}{2.56}a_x  = 2.34\cdot a_x   
  \end{align*}
  The normal components at the boundary are not continuous and therefore we must do more.\\
  The normal is $\hat a_z$\\
  $\mathbf{\hat n}\cdot (\bm{\mathcal{D}}_2 - \bm{\mathcal{D}}_1) = q_{es}$\\
  \begin{align*}
    \hat a_z\cdot (\bm{\mathcal{D}}_2 - \bm{\mathcal{D}}_1) &= 0.2 \\
     \bm{\mathcal{D}}_{1z} = \bm{\mathcal{D}}_{2z} - 0.2 &= 3-.2 = 2.8\\
  \end{align*}
\item[(d)] Electric field intensity inside the air medium.\\ 
  Again using the expression $\bm{\mathcal{D}} = \epsilon \bm{\mathcal{E}}$
  \begin{align*}
    \bm{\mathcal{D}}_{air} &= \epsilon_{air} \bm{\mathcal{E}}_{air} \\
    \bm{\mathcal{E}}_{air} &= \epsilon_{air}^{-1}\bm{\mathcal{D}}_{air}\\
    &= \epsilon_0^{-1}(2.34\cdot a_x + 2.8\cdot a_z)
  \end{align*}
\item[(e)] Electric polarization vector inside the air medium.\\ 
  Again using $\bm{\mathcal{P}}_e = \epsilon_0\chi_e\bm{\mathcal{E}}$\\
  \begin{align*}
    \bm{\mathcal{P}}_{e_{air}} &= \epsilon_0\chi_{e_{air}}\bm{\mathcal{E}}_{air}\\
                         &= \chi_{e_{air}}\textcolor{red}{\cancelto{\textcolor{black}{1}}{\textcolor{black}{\epsilon_0\epsilon_0^{-1}}}}(2.34\cdot a_x + 2.8\cdot a_z)\\
    \chi_{e_{air}} &= \epsilon_{r_{air}} - 1 = 1-1=0\\
    \therefore \bm{\mathcal{P}}_{e_{air}} = \mathbf{0}
  \end{align*}
\end{itemize}
Leave your answers in terms of \( \epsilon_0, \mu_0 \).

% Problem 1.13
\section*{Problem 1.13}
The instantaneous magnetic flux density in free space is given by:
\[
{\bm{\mathcal{B}}} = \hat{a}_x B_x \cos(2y) \sin(\omega t - \pi z) + \hat{a}_y B_y \cos(2x) \cos(\omega t - \pi z)
\]
where \( B_x \) and \( B_y \) are constants. Assuming there are no sources at the observation points \( x, y \), determine the electric displacement current density.\\
We can obtain the electric displacement current $\mathcal{J}_d$ by using the following expressions:
\begin{align}
  \nabla \times \bm{H} &= \bm{J}_i +  \bm{J}_c +  \bm{J}_d\label{h_current}\\
  \bm{B} &= \mu \bm{H}\label{h_b_relation}\\
  \bm{J}_d &= \frac{\partial\bm{D}}{\partial t}\label{displacement_curr}
\end{align}
\newpage
\noindent
We take $\bm{J}_i +  \bm{J}_c$ to be 0 since we have no sources at the observation points. Using expression \ref{h_current} and solving expression \ref{h_b_relation} for $\bm{H}$:
\begin{align*}
  \nabla \times \bm{H} &= \bm{J}_d\label{h_current}\\
  \bm{J}_d &= \nabla \times \mu^{-1}\bm{B}\\
  \bm{J}_d &= \mu^{-1}\nabla \times \bm{B}\\
  &= \mu^{-1} \left[\hat a_x\left(\frac{\partial\bm{B}_z}{\partial y} - \frac{\partial\bm{B}_y}{\partial z}\right) - \hat a_y\left( \frac{\partial\bm{B}_z}{\partial x} - \frac{\partial\bm{B}_x}{\partial z}  \right) + \hat a_z\left( \frac{\partial\bm{B}_y}{\partial x} -  \frac{\partial\bm{B}_x}{\partial y}\right)\right]\\
  \bm{B}_z &= 0\\
  &= \mu^{-1}\left[\hat a_x\left(0 - \frac{\partial\bm{B}_y}{\partial z}\right) - \hat a_y\left( 0 - \frac{\partial\bm{B}_x}{\partial z}  \right) + \hat a_z\left( \frac{\partial\bm{B}_y}{\partial x} -  \frac{\partial\bm{B}_x}{\partial y}\right)\right]\\
  &= \mu^{-1}\left[-\hat a_x \frac{\partial\bm{B}_y}{\partial z} + \hat a_y \frac{\partial\bm{B}_x}{\partial z} + \hat a_z\left(\frac{\partial\bm{B}_y}{\partial x} -  \frac{\partial\bm{B}_x}{\partial y}\right)\right]\\
  \frac{\partial\bm{B}_x}{\partial z} & = -\pi B_x \cos(2y) \cos(\omega t - \pi z)\\
  \frac{\partial\bm{B}_y}{\partial z} & = \pi B_y \cos(2x) \sin(\omega t - \pi z)\\
  \frac{\partial\bm{B}_y}{\partial x} & = -2\sin(2x)cos(\omega t - \pi z)\\
  \frac{\partial\bm{B}_x}{\partial y} & = -2\sin(2y)sin(\omega t - \pi z)\\
  &= \mu^{-1} \left[\hat a_x(- \pi B_y \cos(2x) \sin(\omega t - \pi z) ) \\
  &+ \hat a_y (-\pi B_x \cos(2y) \cos(\omega t - \pi z))\\
  &\hat a_z 2(\sin(2y)sin(\omega t - \pi z) - \sin(2x)cos(\omega t - \pi z))] \\
\end{align*}


% Problem 1.20
\section*{Problem 1.20}
The instantaneous electric field inside a conducting rectangular pipe (waveguide) of width \( a \) is given by:
\[
\bm{\mathcal{E}} = \hat{a}_y E_0 \sin\left(\frac{\pi x}{a}\right) \cos(\omega t - \beta_z z)
\]
where \( \beta_z \) is the waveguide's phase constant. Assuming there are no sources within the free-space-filled pipe determine the:
\begin{itemize}
\item[(a)] Corresponding instantaneous magnetic field components inside the conducting pipe. ($\bm{\mathcal{H}}$)\\
\item[(b)] Phase constant \( \beta_z \).
\end{itemize}
The height of the waveguide is \( b \).
This can be found by relating $\bm{\mathcal{E}}$ to $\bm{\mathcal{H}}$ and we know how to get there using the instantaneous forms of these:
\begin{align*}
  \bm{\mathcal{E}} &= -j \omega\mu\nabla \times \bm{\mathcal{H}} \\
  \bm{\mathcal{H}} &= \frac{j}{\omega\mu}\nabla \times \bm{\mathcal{E}}
\end{align*}
\begin{align*}
  \frac{j}{\omega\mu}  \nabla \times \bm{\mathcal{E}} &=  \frac{j}{\omega\mu} \left[\hat a_x\left(\frac{\partial\bm{E}_z}{\partial y} - \frac{\partial\bm{E}_y}{\partial z}\right) - \hat a_y\left( \frac{\partial\bm{E}_z}{\partial x} - \frac{\partial\bm{E}_x}{\partial z}  \right) + \hat a_z\left( \frac{\partial\bm{E}_y}{\partial x} -  \frac{\partial\bm{E}_x}{\partial y}\right)\right]
\end{align*}

\begin{align*}
  \frac{j}{\omega\mu}  \nabla \times \bm{\mathcal{E}} &=  \frac{j}{\omega\mu} \left[\hat a_x\left(0 - \frac{\partial\bm{E}_y}{\partial z}\right) - \hat a_y\left( 0 - 0  \right) + \hat a_z\left( \frac{\partial\bm{E}_y}{\partial x} -  0\right)\right] \\
  \frac{j}{\omega\mu}  \nabla \times \bm{\mathcal{E}} &=  -\frac{j}{\omega\mu} \left[\hat a_x \frac{\partial\bm{E}_y}{\partial z} - \hat a_z \frac{\partial\bm{E}_y}{\partial x}\right]\\
  \bm{\mathcal{E}} &= \hat{a}_y E_0 \sin\left(\frac{\pi x}{a}\right) \cos(\omega t - \beta_z z)\\
  \frac{\partial\bm{E}_y}{\partial x} &= \left( \frac{\pi}{a}\right) E_0 \cos\left(\frac{\pi x}{a}\right)\cos(\omega t - \beta_z z) \\
  \frac{\partial\bm{E}_y}{\partial z} &= \left( \beta_z\right) E_0 \sin\left(\frac{\pi x}{a}\right)\sin(\omega t - \beta_z z) \\
  \bm{\mathcal{H}} &= -\frac{j}{\omega\mu} \left[\hat a_x \left( \beta_z\right) E_0 \sin\left(\frac{\pi x}{a}\right)\sin(\omega t - \beta_z z) - \hat a_z \left( \frac{\pi}{a}\right) E_0 \cos\left(\frac{\pi x}{a}\right)\cos(\omega t - \beta_z z)\right]
\end{align*}
Solving for $\beta_z$, we can use first expression for the $\nabla \times \bm{\mathcal{H}}$:
\begin{align*}
  \nabla \times \bm{\mathcal{H}} &= (-j \omega\mu)^{-1} \bm{\mathcal{E}}\\
  \nabla \times \bm{\mathcal{H}} &= \left[\hat a_x\left(\frac{\partial\bm{H}_z}{\partial y} - \frac{\partial\bm{H}_y}{\partial z}\right) - \hat a_y\left( \frac{\partial\bm{H}_z}{\partial x} - \frac{\partial\bm{H}_x}{\partial z}  \right) + \hat a_z\left( \frac{\partial\bm{H}_y}{\partial x} -  \frac{\partial\bm{H}_x}{\partial y}\right)\right]
\end{align*}
\newpage
The expression for $\bm{\mathcal{H}}$ we found does not have any dependence on $y$, and only has $x$ and $z$ components:
\begin{align*}
  \nabla \times \bm{\mathcal{H}} &= \left[\hat a_x\left(0 - 0\right) - \hat a_y\left( \frac{\partial\bm{H}_z}{\partial x} - \frac{\partial\bm{H}_x}{\partial z}  \right) + \hat a_z\left( 0 -  0\right)\right]\\
                                 &= - \hat a_y\left( \frac{\partial\bm{H}_z}{\partial x} - \frac{\partial\bm{H}_x}{\partial z}  \right)\\
                                 &= -\frac{j}{\omega\mu} \hat a_y\left(\left(\frac{\pi}{a}\right)^2 E_0\sin{\left(\frac{\pi x}{a}\right)}\cos{(\omega t - \beta_zz)} - \beta_z^2E_0\sin{\left(\frac{\pi x}{a}\right)}\cos{(\omega t - \beta_zz)}  \right)\\
                                 &= \hat a_y\frac{j}{\omega\mu}\left(\beta_z^2 - \left(\frac{\pi}{a}\right)^2\right)E_0\sin{\left(\frac{\pi x}{a}\right)}\cos{(\omega t - \beta_zz)} = \hat a_y\frac{j}{\omega\mu}\left(\beta_z^2 - \left(\frac{\pi}{a}\right)^2\right) \bm{\mathcal{E}}\\
  \therefore \omega \mu &= \frac{1}{\omega\mu}\left(\beta_z^2 - \left(\frac{\pi}{a}\right)^2\right)\\
  \beta_z^2  &= (\omega\mu)^2  + \left(\frac{\pi}{a}\right)^2\\
  \beta_z &= \pm \sqrt{(\omega\mu)^2  + \left(\frac{\pi}{a}\right)^2}
\end{align*}

% Problem 2.18
\section*{Problem 2.18}
The time-varying electric field inside a lossless dielectric material of polystyrene, of infinite dimensions and with a relative permittivity (dielectric constant) of 2.56, is given by:
\[
\vec{\mathcal{E}} = \hat{a}_z 10^{-3} \sin(2\pi \times 10^7 t) \, \text{V/m}
\]
Determine the corresponding:
\begin{itemize}
\item[(a)] Electric susceptibility of the dielectric material.
  \begin{align*}
    \chi_e = \epsilon_r - 1 = 1.56
  \end{align*}
\item[(b)] Time-harmonic electric flux density vector.
  \begin{align*}
\bm{D} &= \epsilon\bm{E} = \hat{a}_z  2.56\times10^{-3} \epsilon_0\sin(2\pi \times 10^7 t)
  \end{align*}
\item[(c)] Time-harmonic electric polarization vector.
    \begin{align*}
      \bm{P} &= \chi_e\epsilon_0\bm{E} = \hat{a}_z  1.56\times10^{-3} \epsilon_0\sin(2\pi \times 10^7 t)
  \end{align*}
\item[(d)] Time-harmonic displacement current density vector.
      \begin{align*}
        \bm{J}_d &= \frac{\partial \bm{D}}{\partial t} = \hat{a}_z  2.56\times10^{-3}\times2\pi \times 10^7 \epsilon_0\cos(2\pi \times 10^7 t) \\
                 &=  \hat{a}_z1.65\times 10^5\epsilon_0\cos(2\pi \times 10^7 t)
  \end{align*}
\item[(e)] Time-harmonic polarization current density vector defined as the partial derivative of the corresponding electric polarization vector.
        \begin{align*}
          \bm{J}_p &= \frac{\partial \bm{P}}{\partial t} = \hat{a}_z  1.56\times10^{-3}\times 2\pi \times 10^7 \epsilon_0\cos(2\pi \times 10^7 t) \\
                   &= \hat{a}_z 9.80\times 10^4 \epsilon_0\cos(2\pi \times 10^7 t)
  \end{align*}
\end{itemize}
Leave your answers in terms of \( \epsilon_0, \mu_0 \).

% Problem 2.25
\section*{Problem 2.25}
Aluminum has a static conductivity of about \( \sigma = 3.96 \times 10^7 \, \text{S/m} \) and an electron mobility of \( \mu_e = 2.2 \times 10^{-3} \, \text{m}^2/(\text{V-s}) \). Assuming that an electric field of \( \vec{E} = \hat{a}_x 2 \, \text{V/m} \) is applied perpendicularly to the square area of an aluminum wafer with cross-sectional area of about \( 10 \, \text{cm}^2 \), find the:\\
 Here are a few things that will be useful in this set of problems:
  \begin{align*}
    v_e &= -\mu_e\bm{E}\\
    \bm{J} &= -q_{ve}\mu_e\bm{E}\\
    \sigma_s &= -q_{ve}\mu_e\\
    q_{ve} &= N_eq_e
  \end{align*}
\begin{itemize}
\item[(a)] Electron charge density \( q_{\text{ve}} \).
   \begin{align*}
    \sigma_s &= -q_{ve}\mu_e\\
    q_{ve} &= -\sigma_s\mu_e^{-1} = -3.96 \times 10^7(2.2 \times 10^{-3})^{-1} = -\frac{3.96}{2.2}\times 10^{10} \left[\frac{C}{m^3}\right] 
  \end{align*}
 
\item[(b)] Electron drift velocity \( v_e \).
  \begin{align*}
    v_e &= -\mu_e\bm{E}\\
    v_e &= -2.2 \times 10^{-3}\hat{a}_x 2 = -\hat{a}_x 4.4\times 10^{-3}  \left[\frac{m}{s}\right]
  \end{align*}
\item[(c)] Electric current density \( J \).
  \begin{align*}
    \bm{J} &= -q_{ve}\mu_e\bm{E}\\
           &= --\frac{3.96}{2.2}\times 10^{10}\times 2.2 \times 10^{-3} \hat{a}_x \times 2 \\
           &= \frac{3.96}{\cancelto{1}{2.2}}\times 10^{\cancelto{7}{10}}\times \cancelto{1}{2.2} \cancel{\times 10^{-3}} \hat{a}_x \times 2\\
   &= 2\times3.96\times 10^7\hat{a}_x = \hat{a}_x 7.92 \times 10^7  \left[\frac{A}{m^2}\right]
  \end{align*}
\item[(d)] Electric current flowing through the square cross section of the wafer.
  \begin{align*}
    I &= \iint_{0}^{A}\bm{J}dA = \bm{J}\iint_{0}^{A}1dA = \bm{J}A = 10cm^2 \times 7.92\times 10^7\hat{a}_x\\
    &= 10^{-3} \times 7.92\times 10^7\hat{a}_x = \hat{a}_x 7.92\times 10^4[A]
  \end{align*}
\item[(e)] Electron density \( N_e \).
  \begin{align*}
    q_{ve} &= N_eq_e\\
    q_e &= 1.6\times 10^{-19}\\
    \therefore N_e &= \left|\frac{q_{ve}}{q_{e}}\right| = \frac{3.96}{2.2 \cdot 1.6}\times 10^{10} \times 10^{19} = 1.125 \times 10^{29}\left[\frac{e}{m^2}\right]
  \end{align*}
\end{itemize}
Leave your answers in terms of \( \epsilon_0, \mu_0 \).


% References (if any)
\bibliographystyle{plain}
\bibliography{references/references}

\end{document}
%%% Local Variables:
%%% mode: latex
%%% TeX-master: t
%%% TeX-master: t
%%% End:
