\documentclass[12pt]{article}
\usepackage{graphicx}
\usepackage{xcolor}
\usepackage{hyperref}
\usepackage{wasysym}
\usepackage{mathrsfs}
\usepackage[calc]{datetime2}
\usepackage{boondox-cal}
\usepackage{bm} % For bold math symbols
\usepackage{amsmath}
\usepackage{amssymb}
\usepackage{cancel}
\usepackage{tikz}
\usetikzlibrary{arrows}

\DTMsavenow{now}
\DTMsavedate{DueDate}{2024-02-21}

\newcount\daystilldue
\newcount\hourstilldue
\newcount\minutestilldue

% Calculate the difference in days
\DTMsaveddatediff{DueDate}{now}{\daystilldue}

% Calculate the difference in hours and minutes
\hourstilldue=\DTMfetchhour{DueDate}
\advance\hourstilldue by -\DTMfetchhour{now}
\minutestilldue=\DTMfetchminute{DueDate}
\advance\minutestilldue by -\DTMfetchminute{now}

% Adjust for negative values
\ifnum\hourstilldue<0
  \advance\hourstilldue by 24
  \advance\daystilldue by -1
\fi
\ifnum\minutestilldue<0
  \advance\minutestilldue by 60
  \advance\hourstilldue by -1
\fi

\newcommand{\TimeUntilDue}{
  \ifnum\daystilldue<4
    \textcolor{red}{
    \number\daystilldue\ days - 
    \number\hourstilldue\ hours - 
    \number\minutestilldue\ min until deadline!!
  }
\else
    \number\daystilldue\ days - 
    \number\hourstilldue\ hours - 
    \number\minutestilldue\ min until deadline
  \fi
}



% Save the original \BibTeX command in case you need it later
\let\originalBibTeX\BibTeX

% Redefine the \BibTeX command to display the LaTeX logo
\renewcommand{\BibTeX}{{\rmfamily B\kern-.03em{\sffamily ib}\kern-.15em\TeX}}


\newcommand{\delCross}[1]{
  \left[\hat a_x\left(\frac{\partial\bm{#1}_z}{\partial y} - \frac{\partial\bm{#1}_y}{\partial z}\right) - \hat a_y\left( \frac{\partial\bm{#1}_z}{\partial x} - \frac{\partial\bm{#1}_x}{\partial z}  \right) + \hat a_z\left( \frac{\partial\bm{#1}_y}{\partial x} -  \frac{\partial\bm{#1}_x}{\partial y}\right)\right]
}
\begin{document}

\newcommand{\Cross}[2]{
\hat a_x(#1_2#2_3 -#1_3#2_2) -\hat a_y(#1_1#2_3-#1_3#2_1) + \hat a_z(#1_1#2_2-#1_2#2_1) 
}
\title{ECE 6310 - Advanced Electromagnetic Fields: Homework Set \#3}
\author{Miguel Gomez}
%\date{\TimeUntilDue}
\maketitle

%\begin{center}
%  $\mathcal{ABCDEFGHIJKLMNOPQRSTUVWXYZ}$\\
%$\mathscr{ABCDEFGHIJKLMNOPQRSTUVWXYZ}$
%\end{center}
\section*{Preliminaries}
In this document, we use standard notation for electromagnetic theory. Key equations and concepts are summarized below:

\subsection*{Vector Notation}
\begin{itemize}
  \item $\bm{\mathcal{E}}$: Electric field intensity
  \item $\bm{\mathcal{H}}$: Magnetic field intensity
  \item $\bm{\mathcal{D}}$: Electric flux density
  \item $\bm{\mathcal{B}}$: Magnetic flux density
  \item $\bm{\mathcal{J}}$: Current density
  \item $\rho_v$: Volume charge density
\end{itemize}

\subsection*{Differential Operators}
\begin{itemize}
  \item $\nabla \cdot\ \ $: Divergence of a vector field
  \item $\nabla \times\ \ $: Curl of a vector field
  \item $\nabla\ \ $: Gradient of a scalar field
  \item $\partial_i\ \ $: Partial derivative with respect to the independent basis element $i$
\end{itemize}

\subsection*{Maxwell's Equations}
In integral form, Maxwell's equations are given by:
\begin{align}
  \oint_{\partial V} \bm{\mathcal{E}} \cdot d\bm{\mathcal{l}} &= - \frac{d}{dt} \int_{V} \bm{\mathcal{B}} \cdot d\bm{\mathcal{S}} & \text{(Faraday's Law of Induction)} \\
  \oint_{\partial V} \bm{\mathcal{H}} \cdot d\bm{\mathcal{l}} &= \int_{V} \bm{\mathcal{J}} \cdot d\bm{\mathcal{S}} + \frac{d}{dt} \int_{V} \bm{\mathcal{D}} \cdot d\bm{\mathcal{S}} & \text{(Ampère's Circuital Law)} \\
  \oiint_{\partial V} \bm{\mathcal{D}} \cdot d\bm{\mathcal{S}} &= \int_{V} \rho_v dV & \text{(Gauss's Law for Electricity)} \\
  \oiint_{\partial V} \bm{\mathcal{B}} \cdot d\bm{\mathcal{S}} &= 0 & \text{(Gauss's Law for Magnetism)}
\end{align}

\subsection*{Other Relevant Equations}
\begin{itemize}
  \item Continuity Equation: $\nabla \cdot \bm{\mathcal{J}} + \partial_t \rho_v = 0$
  \item Relationship between $\bm{\mathcal{E}}$, $\bm{\mathcal{D}}$: $\bm{\mathcal{D}} = \epsilon \bm{\mathcal{E}}$
  \item Relationship between $\bm{\mathcal{H}}$, $\bm{\mathcal{B}}$: $\bm{\mathcal{B}} = \mu \bm{\mathcal{H}}$
\end{itemize}

\subsection*{Boundary Conditions}
Discuss the boundary conditions for $\bm{\mathcal{E}}$, $\bm{\mathcal{H}}$, $\bm{\mathcal{D}}$, and $\bm{\mathcal{B}}$ at interfaces between different media.
\newpage
\section{- 6.19}
Show that for observations made at very large distances $(\beta r \gg 1)$ the electric and magnetic fields of Example 6-3 reduce to the following:
\begin{center}
  \begin{align*}
    E_{\theta} &= \\
    H_{\phi} &\approx \\
    E_{r} &\approx 0\\
    E_{\phi} = H_r &= H_{\theta} = 0
  \end{align*}
\end{center}


\section{- 6.20}


\section{- 6.25}


\section{- 6.26}


\section{- 6.38}


\section{- 7.3}


\section{- 7.15}


\section{- 7.37}

% References (if any)
\bibliographystyle{plain}
\bibliography{references/references}

\end{document}
