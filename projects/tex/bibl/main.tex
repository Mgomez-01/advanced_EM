\documentclass[journal]{IEEEtran}

% Bibliography stuff

% removed the bib package and call below to adhere to the IEEEtran bib styl
%\useepackage{biblatex} %Imports biblatex package
%\addbibresource{bib/citations.bib} %Import the bibliography file

\usepackage[autostyle=false, style=english]{csquotes}
\MakeOuterQuote{"}
\usepackage{dirtytalk}
\usepackage{comment}
\usepackage{ulem}
\usepackage{lscape}
\usepackage{mathrsfs}
\usepackage{cancel}
\usepackage{amsmath,amsthm,enumitem,amssymb}
\usepackage{appendix}
\usepackage{xcolor}
\usepackage{circuitikz}
\usepackage{tikz}
\usepackage{tikz-3dplot}
\usepackage{caption}
\usepackage{hyperref}\usepackage{listings}
\usepackage{fancyvrb}
\usepackage{framed}
\usepackage[listings,skins]{tcolorbox}
\usepackage[skipbelow=\topskip,skipabove=\topskip]{mdframed}
\mdfsetup{roundcorner=1}

\usetikzlibrary{arrows.meta,calc}



% Fancy
%%%%%%%%%
\definecolor{MyComments}{HTML}{55CC10}
\definecolor{MyKeywords}{HTML}{1E47FC}
\definecolor{MyStrings}{HTML}{A2A8C1}
\definecolor{MyIdentifiers}{HTML}{000000}

\lstset{
  backgroundcolor=\color{white},
  basicstyle=\footnotesize\ttfamily,
  breaklines=true,
  captionpos=t,
  commentstyle=\color{MyComments},
  keywordstyle=\color{MyKeywords},
  stringstyle=\color{MyStrings},
  identifierstyle=\color{MyIdentifiers}
}

%%%%%%%%%%
\usepackage{caption}
% *** GRAPHICS RELATED PACKAGES ***
%
\ifCLASSINFOpdf
   \usepackage[pdftex]{graphicx}
  % declare the path(s) where your graphic files are
  % \graphicspath{{../pdf/}{../jpeg/}}
  % and their extensions so you won't have to specify these with
  % every instance of \includegraphics
  % \DeclareGraphicsExtensions{.pdf,.jpeg,.png}
\else
  % or other class option (dvipsone, dvipdf, if not using dvips). graphicx
  % will default to the driver specified in the system graphics.cfg if no
  % driver is specified.
  % \usepackage[dvips]{graphicx}
  % declare the path(s) where your graphic files are
  % \graphicspath{{../eps/}}
  % and their extensions so you won't have to specify these with
  % every instance of \includegraphics
  % \DeclareGraphicsExtensions{.eps}
\fi

\newcommand{\figref}[1]{[\ref{#1}]}

\begin{document}
\title{Reference page for the Exploration of Geometric Algebra for Microwave Network Classification, Simplification, and self-study }

\author{
    \IEEEauthorblockN{Miguel Gomez}\\
    \IEEEauthorblockA{\textit{University of Utah Computer Engineering}}
    }

\maketitle

\IEEEpeerreviewmaketitle
\begin{abstract}
  \begin{IEEEkeywords}
Bivector,
Clifford Algebra,
Geometric Algebra, 
Mobius Transformation,
Multivector, 
Rotor, 
Spinor
\end{IEEEkeywords}

\end{abstract}


\section{Introduction}
I am adding in all my references here for the assignment. 
  \cite{Montoya_9248978}
  \cite{Montoya_9808623}
  \cite{Zhang_9943322}
  \cite{Li_9913488}
  \cite{Franchini_6783758}
  \cite{Eid_9992202}
  \cite{Corrochano_9488174}
  \cite{Neve_10345560}
  \cite{Arsenovic_7982607}
  \cite{scikit_9632487}
\section{Reflection}
\subsection{My Process}
\noindent
My journey has had me working on papers and compiling sources for some time now. I do not like using Mendeley as they broke the integration with Google Chromium based browsers and have yet to do anything to fix it. I used it before and it was useful, but I have taken to keeping the PDFs of the papers that I put together, and I always get the bibtex entry from the source itself. I add them all to a custom bibliography that contains everything which is backed up on Github as well as my home PC and Laptop. That way I am never too far away from my sources and have an easy method of sharing them with others.\\
Additionally, I have been using Emacs as my main text editor for some time since it is very easy to have custom code that runs when I am working in a particular language. One of which is the \LaTeX $\ $ language for markup. It is easy to put together papers with the exact formatting that I want, or in the case of the main place where I am looking to publish, IEEE has the IEEETran document style that is provided for their journals.  
\subsection{Things I liked this week}
\noindent
I had not used the Open Knowledge Map before but I really liked the way things are set up there. Knowing that I could easily take the topic that I am working on and create a map like this by just using a single search term is a powerful tool. It was very rewarding and validating to see many of the same resources that I have come across show up in the list of papers and topic bubbles when looking at Geometric Algebra. I will likely continue to use these in the future. I particularly liked the fact that the bubbles contained papers that likely had a smaller Euclidean distance in the N-dimensional space where this data is compiled.
\bibliographystyle{IEEEtran}
\bibliography{IEEEabrv,GA_reference.bib}
%\printbibliography
% that's all folks
\end{document}

%%% Local Variables:
%%% mode: latex
%%% TeX-master: t
%%% End:
